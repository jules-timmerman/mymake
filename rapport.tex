\documentclass[11pt]{article}
\usepackage[utf8]{inputenc}
\usepackage[french]{babel}
\usepackage[T1]{fontenc}
\usepackage{mathrsfs}

\usepackage[a4paper,width=160mm,top=35mm,bottom=35mm]{geometry}

\usepackage{booktabs}
\usepackage{array}
\usepackage{subfig}
\usepackage{amsmath,amsfonts,amssymb}

\usepackage{mathtools}
\usepackage{titlesec} 
\usepackage{enumitem}
\usepackage{listings}
\usepackage{mathdots}
\usepackage{comment}


% Codes source
\usepackage{listings}

% Encadrés
\usepackage{tcolorbox}


%\renewcommand{\thesection}{\arabic{section}}
%\renewcommand{\thesubsection}{\arabic{subsection}}


\usepackage{parskip}
\usepackage{tgpagella}


\newcounter{question}[section]
\newenvironment{question}[1][]{\refstepcounter{question}\par\medskip
   \noindent\textbf{Question~\thequestion ~ $-$} \rmfamily}{}


\newtcolorbox[auto counter,number within=section]{code}[2][]{
fonttitle=\bfseries,
title=Code source de #2,#1,
colback=white,
colframe=black,
arc=0mm
}


\newcommand{\Codesource}[2]{
\begin{code}[]{\texttt{ #1 }}
\lstinputlisting{ #2 }
\end{code}

}


\title{Projet MyMake}
\date{Novembre 2022}
\author{Guilhem Repetto \and Jules Timmerman}

\begin{document}

\maketitle

\begin{abstract}
 Lorem ipsum dolor sit amet, consectetur adipiscing elit. Cras consectetur tincidunt magna, eu aliquam velit dapibus tristique. Nullam mattis enim ut eros gravida condimentum. Sed rutrum dictum massa sit amet mattis. Vivamus vehicula elementum mauris. Nam sodales efficitur massa eu volutpat. Nam accumsan est id lectus dictum aliquam. Nunc lobortis molestie arcu vel maximus. Vivamus neque tellus, rhoncus in justo ut, aliquam aliquet est. Suspendisse semper consequat congue. Maecenas et quam eu dolor commodo luctus. Pellentesque habitant morbi tristique senectus et netus et malesuada fames ac turpis egestas. Quisque nec diam ac sapien luctus pharetra sit amet quis arcu. Ut ut nisi malesuada, rhoncus tortor non, porttitor neque. Duis maximus ornare quam, id dictum nulla tincidunt et.
\end{abstract}




\begin{question}
Pour créer le fichier \texttt{Makefile}, analysons la dépendance entre les fichiers du projet :

\Codesource{Makefile}{testproj/Makefile}




\end{question}

\begin{question}

\end{question}




\begin{thebibliography}{99}

\bibitem{michou}
  \textit{Thèmes d'analyse (p.139 - 165)},
  Jean-Marie Exbrayat, Michel Alessandri,
  Masson,
  1997.

\bibitem{troisdemis}
  \textit{La suite des puissances de 3/2},
  F.D. et M.M.F.,
  La Recherche,
  2001.


\end{thebibliography}






\end{document}