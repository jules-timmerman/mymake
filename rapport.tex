\documentclass[11pt]{report}
\usepackage[utf8]{inputenc}
\usepackage[french]{babel}
\usepackage[T1]{fontenc}
\usepackage{mathrsfs}

\usepackage[a4paper,width=160mm,top=35mm,bottom=35mm]{geometry}

\usepackage{booktabs}
\usepackage{array}
\usepackage{subfig}
\usepackage{amsmath,amsfonts,amssymb}

\usepackage{mathtools}
\usepackage{titlesec} 
\usepackage{enumitem}
\usepackage{listings}
\usepackage{mathdots}
\usepackage{comment}


% Codes source
\usepackage{listings}
\usepackage{xcolor}

\definecolor{codegreen}{rgb}{0,0.6,0.1}
\definecolor{codegray}{rgb}{0.5,0.5,0.5}
\definecolor{codepurple}{rgb}{0.58,0,0.82}
\definecolor{backcolour}{rgb}{0.95,0.95,0.92}

\lstdefinestyle{mystyle}{
    backgroundcolor=\color{backcolour},   
    commentstyle=\color{codegreen},
    keywordstyle=\color{blue},
    numberstyle=\tiny\color{codegray},
    stringstyle=\color{codepurple},
    basicstyle=\ttfamily\footnotesize,
    breakatwhitespace=false,         
    breaklines=true,                 
    captionpos=b,                    
    keepspaces=true,                 
    numbers=left,                    
    numbersep=5pt,                  
    showspaces=false,                
    showstringspaces=false,
    showtabs=false,                  
    tabsize=2
}

\lstset{literate=
{é}{{\'e}}{1}
{è}{{\`e}}{1}
{à}{{\`a}}{1}
{ç}{{\c{c}}}{1}
{œ}{{\oe}}{1}
{ù}{{\`u}}{1}
{É}{{\'E}}{1}
{È}{{\`E}}{1}
{À}{{\`A}}{1}
{Ç}{{\c{C}}}{1}
{Œ}{{\OE}}{1}
{Ê}{{\^E}}{1}
{ê}{{\^e}}{1}
{î}{{\^i}}{1}
{ô}{{\^o}}{1}
{û}{{\^u}}{1}
{ë}{{\¨{e}}}1
{û}{{\^{u}}}1
{â}{{\^{a}}}1
{Â}{{\^{A}}}1
{Î}{{\^{I}}}1
} 

\lstset{style=mystyle}

% Encadrés
\usepackage{tcolorbox}

% Graphes
\usepackage{tikz}
\usepackage{tikz-cd}


%\renewcommand{\thesection}{\arabic{section}}
%\renewcommand{\thesubsection}{\arabic{subsection}}


\usepackage{parskip}
\usepackage{tgpagella}


\newcounter{question}[section]
\newenvironment{question}[1][]{\refstepcounter{question}\par\medskip
   \noindent\textbf{Question~\thequestion ~ $-$} \rmfamily}{}


\newtcolorbox[auto counter,number within=section]{code}[2][]{
fonttitle=\bfseries,
title=Code source de #2,#1,
colback=white,
colframe=black,
arc=0mm
}





\title{MyMake}
\date{Octobre 2022}
\author{Guilhem Repetto \and Jules Timmerman}


\begin{document}

\maketitle

\begin{abstract}

\end{abstract}



\begin{question} %% Question 1
Pour créer le fichier \textbf{Makefile}, il faut comprendre la dépendance entre les fichiers du projet.
Nous voulons compiler un fichier exécutable \textbf{main} à partir du fichier \textbf{main.c}.

Tout d'abord, rappelons qu'au minimum, le fichier \textbf{main.c} est transformé par le préprocesseur en un fichier \textbf{main.i}, compilé en un fichier assembleur \textbf{main.s}, puis en un fichier binaire \textbf{main.o}. Enfin, l'étape d'édition des liens utilise \textbf{main.o} ainsi que tous les fichiers binaires (en \textbf{.o}) des bibliothèques et des fichiers annexes déclarés dans le \textit{header} de \textbf{main.c} (fichiers en \textbf{.h}) utilisés pour produire un fichier final exécutable \textbf{main}.


Pour produire \textbf{main}, le compilateur a donc besoin du fichier \textbf{main.o}. Dans notre cas, les fichiers \textbf{c.h} et \textbf{d.h} sont déclarés en \textit{header}, ce qui implique que les fichiers \textbf{c.o} et \textbf{d.o} sont aussi demandés. De plus, les fichiers \textbf{a.h} et \textbf{b.h} sont demandés par \textbf{c.h}, et \textbf{a.h} est demandé par \textbf{d.h}. Pour compiler le fichier \textbf{main.c}, il faut donc avoir accès aux fichiers \textbf{main.o}, \textbf{a.o}, \textbf{b.o}, \textbf{c.o} et \textbf{d.o}.
On ajoute donc les lignes suivantes au fichier \textbf{Makefile} :
\begin{verbatim}
main: main.o a.o b.o c.o d.o
    gcc -o main main.o a.o b.o c.o d.o
\end{verbatim}

Pour le reste, on construit alors un graphe de dépendance (voir figure \ref{fig:dependance_test}), où $\mathbf{x}\rightarrow \mathbf{y}$ signifie que le fichier \textbf{y} a besoin du fichier \textbf{x} pour être compilé.


\begin{figure}[h]
\centering
% https://tikzcd.yichuanshen.de/#N4Igdg9gJgpgziAXAbVABwnAlgFyxMJZABgBpiBdUkANwEMAbAVxiRAB12cYAPHf4FAB0AYwC+IMaXSZc+QijIBGKrUYs2nbnwHCAFhKkzseAkTIAmVfWatEHLr345gdUYekgMJ+edIBma3U7B21nVyEDSU9vOTNFUgAWINtNRx0XEXdo4ziFElIAVhSNey0nASyooy9ZU3yyADYSkPKM4AAjbJrY+r8Adha0sIEu6pi63xQLUhVqG1LQipdhCA9cvumAobL08Lc1nNqfeOQZ4vng4eXgLMOeydOZwcvU3ZGXLvuJk-zEpJ2S3aAFs6FgwOtjnkiP8rK9Fm1wqDwUJ7qoYFAAObwIigABmACcIMCkGQQDgIEgABw1QnEpBKagU6m0okkxAzcmUxAATlZ9MQ-iZ3L5njp7P+XIZxH57MKwulsqQjQViCUSiViH6qvVmqFUrVZIY4JCcAgxqgR3FSH1zMQVOoHRgYEtiBlYrZ1J1Fk1PO9mvV-o9AqUZLtSh9wfZSlt3Ij1D0MDorrATAYDE18oNMcdztd7vxnrVOv1TpdpLEFDEQA
\begin{tikzcd}
\textbf{d.c} \arrow[rrd]                          &  &                            &  &                            \\
\textbf{d.h} \arrow[rr] \arrow[rrrrd]             &  & \textbf{d.o} \arrow[rrddd] &  &                            \\
\textbf{a.c} \arrow[rrd]                          &  &                            &  & \textbf{main.o} \arrow[dd] \\
\textbf{a.h} \arrow[rr] \arrow[rrdd] \arrow[rruu] &  & \textbf{a.o} \arrow[rrd]   &  &                            \\
\textbf{c.c} \arrow[rrd]                          &  &                            &  & \textbf{main}              \\
\textbf{c.h} \arrow[rr] \arrow[rrrruuu]           &  & \textbf{c.o} \arrow[rru]   &  &                            \\
\textbf{b.c} \arrow[rrd]                          &  &                            &  &                            \\
\textbf{b.h} \arrow[rr]                           &  & \textbf{b.o} \arrow[rruuu] &  &                           
\end{tikzcd}
\caption{Relations de dépendance entre les fichiers du dossier \textbf{test}}
\label{fig:dependance_test}
\end{figure}

Nous pouvons alors en déduire le fichier \textbf{Makefile} (figure \ref{code:makefile_test}).



\begin{figure}[h]
\lstinputlisting[language=c]{ testproj/Makefile }
\caption{Code source de \textbf{Makefile} du projet de test}
\label{code:makefile_test}
\end{figure}


\end{question}


\newpage


\begin{question} %% Question 2

Nous créons un module \textbf{regles.c}, où l'on définit une structure \texttt{regle}. Une \texttt{regle} contient un pointeur vers une chaîne \texttt{nom}, un pointeur vers un tableau \texttt{prerequis} qui représente l'ensemble des fichiers nécessaires pour compiler [\texttt{nom}]. Il y a aussi \texttt{lenPrerequis} la taille de ce tableau que nous stockons pour plus de simplicité dans le code. La liste des commandes à effectuer est stockée dans la liste (nous définissons les listes chaînées dans le module \textbf{listeRegles}) des \texttt{commandes}. Enfin, nous avons choisi d'implémenter l'extension qui remplace la date de dernière modification du fichier par un hash. Les attributs \texttt{lastModified} (temps) et \texttt{hashModified} (booléen) sont donc utilisés, et leur fonctionnement est détaillé en question 10.


Cette structure est munie des fonctions permettant la création de règle (\texttt{createRegle}) et la suppression (\texttt{freeRegle}). Les autres fonctions permettent la gestion des dernières modifications et des hashs, et sont détaillées en question 10.

\end{question}

\begin{question} %% Question 3

Nous définissons un module \textbf{listeRegles.c} qui permet de gérer les ensembles de règles. Comme le code nécessite de parcourir l'ensemble , nous décidons d'implémenter ces ensembles à l'aide de listes simplement chaînées.
La structure \texttt{listeRegles} est donc le pointeur \texttt{NULL}, ou un couple (pointeur vers une \texttt{regle}, pointeur vers une \texttt{listeRegles}).

Les fonctions \texttt{createListeRegle} et \texttt{freeListeRegle} permettent la création et la suppression d'une \texttt{listeRegles}.

La fonction \texttt{addRegle} permet d'ajouter une \texttt{regle} à une \texttt{listeRegles}, via un ajout à la tête.

La fonction \texttt{rechercheRegle} recherche une \texttt{regle} par son nom dans une \texttt{listeRegles}, et la renvoie si elle existe.

La fonction \texttt{iterRegles} permet d'appliquer une fonction à tous les éléments d'une \texttt{listeRegles}. 

La fonction \texttt{revListRegle} renvoie la liste renversée de celle donnée en argument.

Enfin, les fonctions \texttt{getLatestModify} et \texttt{childModified} seront détaillées en question 10.



\end{question}

\begin{question} %% Question 4

Le fichier \textbf{Makefile} du projet est le suivant :

\begin{figure}[h]
\lstinputlisting{Makefile}
\caption{Code source de \textbf{Makefile} de \texttt{myMake}}
\end{figure}
\end{question}

\begin{question} %% Question 5

Dans le fichier \texttt{main.c}, nous définissons une fonction \texttt{makefile2list}, qui prend en argument un pointeur vers un fichier \textbf{makefile} valide, et renvoie un pointeur vers une liste de règles représentant les règles définies dans le fichier.

Son fonctionnement est le suivant :

\begin{enumerate}
\item une liste vide de règles est créée
\item la ligne courante du fichier est lue, elle est du type \texttt{nom : <listePrerequis>}. Une nouvelle règle vide est crée, où la fonction ajoute toutes les lignes suivantes non-vides, c'est-à-dire des lignes de type \texttt{<commande>}
\item la première ligne vide rencontrée (constituée du seul caractère $\mathtt{\backslash n}$) lue, ou la fin du fichier, déclenche l'ajout de la règle courante dans la liste.
\item Retour à l'étape 2
\end{enumerate}


La lecture s'effectue grâce à la fonction \texttt{getline}. Ainsi, la commande \begin{center}
\texttt{tailleLigne = getline(\&ligne\_buffer, \&tailleLigne\_buffer, makefile)}
\end{center}
écrit la ligne courante du fichier \texttt{makefile} dans la chaîne \texttt{ligne\_buffer}, stocke sa longueur dans \texttt{taille\_ligne} (en comptant le retour à la ligne). La variable \texttt{tailleLigne\_buffer} ne sera pas utilisée ici.




\end{question}

\begin{question} %% Question 6
Pour la fonction \texttt{make\_naive}, nous avons adopté une approche récursive.
\begin{itemize}
	\item \underline{Cas de base} : les fichiers sources (\textbf{.c}) ainsi que les headers (\textbf{.h}) sont des fichiers qui ne nécessitent aucune compilation.
	On peut alors directement finir l'exécution ici.
	\item \underline{Cas général} : On utilise la fonction \texttt{createListeRegleFromPre} qui prend en argument la liste de règle qui a été créée lors de la lecture du \textbf{Makefile} ainsi qu'une règle
	et renvoie une liste de règle constituée des prérequis de la règle passée en argument.

	Cependant, en lisant uniquement les règles depuis la liste du \textbf{Makefile}, on a alors aucune règle dans la liste renvoyée qui représente les fichiers \textbf{.h} et \textbf{.c} puisqu'ils n'ont aucune règle associée.

	On introduit alors le concept de \underline{pseudo-règle}. Il s'agit d'une fausse règle représentant un tel fichier. Il n'est alors pas nécessaire d'initialiser certains attributs comme les prérequis ou les commandes puisqu'il n'y en a pas.
	On crée donc de tels règles lors de l'exécution de \texttt{createListRegleFromPre} et on libérera la mémoire utilisée en même temps que celle de la liste retournée.

	Une fois que l'on a construit la liste des prérequis, on effectue les appels récursifs à l'aide de la fonction \texttt{iterRegles} qui va appliquer la fonction \texttt{make\_naive} avec comme premier argument la liste générale de règles du \textbf{Makefile} à chaque prérequis.

	Après ces appels, on a donc construit nos récursivement les prérequis et l'on peut ensuite construire la règle elle-même.
	
	On libère ensuite la mémoire prise par la liste des prérequis construite ainsi que les pseudo-règles.
\end{itemize}
\end{question}

\begin{question} %% Question 7
La fonction \texttt{make} est presque identique à la fonction \texttt{make\_naive}. On rajoute une condition après nos appels récursifs pour estimer si l'on doit recompiler ou pas.

On récupère la date de modification du fichier liée à la règle considérée ainsi que la dernière date de modification des prérequis avec la fonction \texttt{getLatestModify}.
On utilise l'attribut \texttt{lastModified} de \texttt{regle} pour stocker ces dates. On l'initialise une première fois lors de la construction de la règle. On utilise la librairie \texttt{stat} pour récupérer les dates dans la fonction \texttt{getLastModified}.

Si un des prérequis a été modifié après la règle considérée, cela signifie que le fichier lié n'est pas à jour : il nous faut donc le recompiler et on exécute les commandes. En plus d'exécuter les commandes, on change aussi la date de dernière modification pour permettre de recompiler en cascade lors de la remontée des appels récursifs.

\end{question}




% A rajouter à terme :
% Pointeurs de fonctions en C

\begin{thebibliography}{99}
  \bibitem{getline}
  \textit{Parcourir un fichier texte ligne par ligne en C},
  \texttt{code-examples.net/fr/q/8c794b},
  08/10/22.

  \bibitem{strtok}
  \textit{C library function - strtok()},
  \texttt{tutorialspoint.com/c\_standard\_library/c\_function\_strtok.htm},
  08/10/22.
  
  \bibitem{hash}
  \textit{Hash Functions},
  \texttt{cse.yorku.ca/$\sim$oz/hash.html},
  08/10/22.


\end{thebibliography}

\end{document}