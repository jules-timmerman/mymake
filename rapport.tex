\documentclass[11pt]{article}
\usepackage[utf8]{inputenc}
\usepackage[french]{babel}
\usepackage[T1]{fontenc}
\usepackage{mathrsfs}

\usepackage[a4paper,width=160mm,top=35mm,bottom=35mm]{geometry}

\usepackage{booktabs}
\usepackage{array}
\usepackage{subfig}
\usepackage{amsmath,amsfonts,amssymb}

\usepackage{mathtools}
\usepackage{titlesec} 
\usepackage{enumitem}
\usepackage{listings}
\usepackage{mathdots}
\usepackage{comment}


% Codes source
\usepackage{listings}
\lstset{basicstyle=\ttfamily}

% Encadrés
\usepackage{tcolorbox}

% Graphes
\usepackage{tikz}
\usepackage{tikz-cd}


%\renewcommand{\thesection}{\arabic{section}}
%\renewcommand{\thesubsection}{\arabic{subsection}}


\usepackage{parskip}
\usepackage{tgpagella}


\newcounter{question}[section]
\newenvironment{question}[1][]{\refstepcounter{question}\par\medskip
   \noindent\textbf{Question~\thequestion ~ $-$} \rmfamily}{}


\newtcolorbox[auto counter,number within=section]{code}[2][]{
fonttitle=\bfseries,
title=Code source de #2,#1,
colback=white,
colframe=black,
arc=0mm
}


\newcommand{\Codesource}[2]{
\begin{code}[]{\texttt{ #1 }}
\lstinputlisting{ #2 }
\end{code}

}


\title{MyMake}
\date{Novembre 2022}
\author{Guilhem Repetto \and Jules Timmerman}










\begin{document}

\maketitle

\begin{abstract}
 Lorem ipsum dolor sit amet, consectetur adipiscing elit. Cras consectetur tincidunt magna, eu aliquam velit dapibus tristique. Nullam mattis enim ut eros gravida condimentum. Sed rutrum dictum massa sit amet mattis. Vivamus vehicula elementum mauris. Nam sodales efficitur massa eu volutpat. Nam accumsan est id lectus dictum aliquam. Nunc lobortis molestie arcu vel maximus. Vivamus neque tellus, rhoncus in justo ut, aliquam aliquet est. Suspendisse semper consequat congue. Maecenas et quam eu dolor commodo luctus. Pellentesque habitant morbi tristique senectus et netus et malesuada fames ac turpis egestas. Quisque nec diam ac sapien luctus pharetra sit amet quis arcu. Ut ut nisi malesuada, rhoncus tortor non, porttitor neque. Duis maximus ornare quam, id dictum nulla tincidunt et.
\end{abstract}




\begin{question}
Pour créer le fichier \texttt{Makefile}, il faut comprendre la dépendance entre les fichiers du projet.
Nous voulons compiler un fichier exécutable \texttt{main} à partir du fichier \texttt{main.c}.

Tout d'abord, rappelons qu'au minimum, le fichier \texttt{main.c} est transformé par le préprocesseur en un fichier \texttt{main.i}, compilé en un fichier assembleur \texttt{main.s}, puis en un fichier binaire \texttt{main.o}. Enfin, l'étape d'édition des liens utilise \texttt{main.o} ainsi que tous les fichiers binaires (en \texttt{.o}) des bibliothèques et des fichiers annexes déclarés dans le \textit{header} de \texttt{main.c} (fichiers en \texttt{.h}) utilisés pour produire un fichier final exécutable \texttt{main}.


Pour produire \texttt{main}, le compilateur a donc besoin du fichier \texttt{main.o}. Dans notre cas, les fichiers \texttt{c.h} et \texttt{d.h} sont déclarés en \textit{header}, ce qui implique que les fichiers \texttt{c.o} et \texttt{d.o} sont aussi demandés. De plus, les fichiers \texttt{a.h} et \texttt{b.h} sont demandés par \texttt{c.h}, et \texttt{a.h} est demandé par \texttt{d.h}. Pour compiler le fichier \texttt{main}, il faut donc avoir accès aux fichiers \texttt{main.o}, \texttt{a.o}, \texttt{b.o}, \texttt{c.o} et \texttt{d.o}.
On ajoute donc les lignes suivantes au fichier \texttt{Makefile} :
\begin{verbatim}
main: main.o a.o b.o c.o d.o
    gcc -o main main.o a.o b.o c.o d.o
\end{verbatim}

Pour le reste, on construit alors un graphe de dépendance (voir figure \ref{fig:dependance_test}), où $\mathtt{x}\rightarrow \mathtt{y}$ signifie que le fichier \texttt{y} a besoin du fichier \texttt{x} pour être compilé.


\begin{figure}
\centering
\begin{tikzcd}
\texttt{a.c} \arrow[rrd]                                                              &  &                    &  &  &                   \\
\texttt{a.h} \arrow[u] \arrow[dddd, bend right] \arrow[dddddd, bend right] \arrow[rr] &  & \texttt{a.o} \arrow[rrrddd] &  &  &                   \\
\texttt{b.c} \arrow[rrd]                                                              &  &                    &  &  &                   \\
\texttt{b.h} \arrow[u] \arrow[dd, bend right] \arrow[rr]                              &  & \texttt{b.o} \arrow[rrrd]   &  &  &                   \\
\texttt{c.c} \arrow[rrd]                                                              &  &                    &  &  & \texttt{main}              \\
\texttt{c.h} \arrow[u] \arrow[rr]                                                     &  & \texttt{c.o} \arrow[rrru]   &  &  &                   \\
\texttt{d.c} \arrow[rrd]                                                              &  &                    &  &  & \texttt{main.o} \arrow[uu] \\
\texttt{d.h} \arrow[u] \arrow[rr]                                                     &  & \texttt{d.o} \arrow[rrruuu] &  &  &                  
\end{tikzcd}
\caption{Relations de dépendance entre les fichiers du dossier \texttt{test}}
\label{fig:dependance_test}
\end{figure}

On peut alors en déduire le fichier \texttt{Makefile}.


\Codesource{Makefile}{testproj/Makefile}




\end{question}

\begin{question}

\end{question}




\begin{thebibliography}{99}

\bibitem{michou}
  \textit{Thèmes d'analyse (p.139 - 165)},
  Jean-Marie Exbrayat, Michel Alessandri,
  Masson,
  1997.

\bibitem{troisdemis}
  \textit{La suite des puissances de 3/2},
  F.D. et M.M.F.,
  La Recherche,
  2001.


\end{thebibliography}






\end{document}