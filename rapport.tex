\documentclass[11pt]{article}
\usepackage[utf8]{inputenc}
\usepackage[french]{babel}
\usepackage[T1]{fontenc}
\usepackage{mathrsfs}

\usepackage[a4paper,width=160mm,top=35mm,bottom=35mm]{geometry}

\usepackage{booktabs}
\usepackage{array}
\usepackage{subfig}
\usepackage{amsmath,amsfonts,amssymb}

\usepackage{mathtools}
\usepackage{titlesec} 
\usepackage{enumitem}
\usepackage{listings}
\usepackage{mathdots}
\usepackage{comment}

\usepackage{algorithm}
\usepackage{algpseudocode}


% Codes source
\usepackage{listings}
\usepackage{xcolor}

\definecolor{codegreen}{rgb}{0,0.6,0.1}
\definecolor{codegray}{rgb}{0.5,0.5,0.5}
\definecolor{codepurple}{rgb}{0.58,0,0.82}
\definecolor{backcolour}{rgb}{0.8,0.8,0.8}

\lstdefinestyle{mystyle}{
    backgroundcolor=\color{backcolour},   
    commentstyle=\color{codegreen},
    keywordstyle=\color{blue},
    numberstyle=\tiny\color{codegray},
    stringstyle=\color{codepurple},
    basicstyle=\ttfamily\footnotesize,
    breakatwhitespace=false,         
    breaklines=true,                 
    captionpos=b,                    
    keepspaces=true,                 
    numbers=left,                    
    numbersep=5pt,                  
    showspaces=false,                
    showstringspaces=false,
    showtabs=false,                  
    tabsize=2
}

\lstset{literate=
{é}{{\'e}}{1}
{è}{{\`e}}{1}
{à}{{\`a}}{1}
{ç}{{\c{c}}}{1}
{œ}{{\oe}}{1}
{ù}{{\`u}}{1}
{É}{{\'E}}{1}
{È}{{\`E}}{1}
{À}{{\`A}}{1}
{Ç}{{\c{C}}}{1}
{Œ}{{\OE}}{1}
{Ê}{{\^E}}{1}
{ê}{{\^e}}{1}
{î}{{\^i}}{1}
{ô}{{\^o}}{1}
{û}{{\^u}}{1}
{ë}{{\¨{e}}}1
{û}{{\^{u}}}1
{â}{{\^{a}}}1
{Â}{{\^{A}}}1
{Î}{{\^{I}}}1
}

\renewcommand{\algorithmicrequire}{\textbf{Require:}}
\renewcommand{\algorithmicensure}{\textbf{Ensure:}}
\renewcommand{\algorithmicend}{\textbf{fin}}
\renewcommand{\algorithmicif}{\textbf{si}}
\renewcommand{\algorithmicthen}{\textbf{alors}}
\renewcommand{\algorithmicelse}{\textbf{sinon}}
\newcommand{\algorithmicelsif}{\algorithmicelse\ \algorithmicif}
\newcommand{\algorithmicendif}{\algorithmicend\ \algorithmicif}
\renewcommand{\algorithmicfor}{\textbf{pour}}
\renewcommand{\algorithmicforall}{\textbf{pour tout}}
\renewcommand{\algorithmicdo}{\textbf{faire}}
\newcommand{\algorithmicendfor}{\algorithmicend\ \algorithmicfor}
\renewcommand{\algorithmicwhile}{\textbf{tant que}}
\newcommand{\algorithmicendwhile}{\algorithmicend\ \algorithmicwhile}
\renewcommand{\algorithmicloop}{\textbf{loop}}
\newcommand{\algorithmicendloop}{\algorithmicend\ \algorithmicloop}
\renewcommand{\algorithmicrepeat}{\textbf{repeat}}
\renewcommand{\algorithmicuntil}{\textbf{until}}
\newcommand{\algorithmicprint}{\textbf{print}}
\renewcommand{\algorithmicreturn}{\textbf{renvoyer}}
\newcommand{\algorithmictrue}{\textbf{true}}
\newcommand{\algorithmicfalse}{\textbf{false}}

\makeatletter
\renewcommand*{\ALG@name}{Algorithme}
\makeatother

\lstset{style=mystyle}

% Encadrés
\usepackage{tcolorbox}

% Graphes
\usepackage{tikz}
\usepackage{tikz-cd}


%\renewcommand{\thesection}{\arabic{section}}
%\renewcommand{\thesubsection}{\arabic{subsection}}


\usepackage{parskip}
\usepackage{tgpagella}


\newcounter{question}[section]
\newenvironment{question}[1][]{\refstepcounter{question}\par\medskip
   \noindent\textbf{Question~\thequestion ~ $-$} \rmfamily}{}


\newtcolorbox[auto counter,number within=section]{code}[2][]{
fonttitle=\bfseries,
title=Code source de #2,#1,
colback=white,
colframe=black,
arc=0mm
}





\title{MyMake}
\date{Octobre 2022}
\author{Guilhem Repetto \and Jules Timmerman}


\begin{document}

\maketitle

\begin{abstract}
Ce projet consiste à faire une implémentation en C de l'application.
Ce programme est un utilitaire permettant de faciliter la compilation d'un programme en spécifiant les différentes étapes intermédiaires par un système de règles décrit dans un fichier appelé \textbf{Makefile}.
Nous implémenterons tout d'abord une fonction permettant la lecture du fichier de règle et la création des structures nécessaire. Ensuite nous construirons notre programme à l'aide des règles imposées.
Implémenter nous-même ce logiciel nous permettra de comprendre son fonctionnement et son utilisation.

\end{abstract}



\begin{question} %% Question 1

Pour créer le fichier \textbf{Makefile}, il faut comprendre la dépendance entre les fichiers du projet.
Nous voulons compiler un fichier exécutable \textbf{main} à partir du fichier \textbf{main.c}.

Tout d'abord, rappelons qu'au minimum, le fichier \textbf{main.c} est transformé par le préprocesseur en un fichier \textbf{main.i}, compilé en un fichier assembleur \textbf{main.s}, puis en un fichier binaire \textbf{main.o}. Enfin, l'étape d'édition des liens utilise \textbf{main.o} ainsi que tous les fichiers binaires (en \textbf{.o}) des bibliothèques et des fichiers annexes déclarés dans le \textit{header} de \textbf{main.c} (fichiers en \textbf{.h}) utilisés pour produire un fichier final exécutable \textbf{main}.


Pour produire \textbf{main}, le compilateur a donc besoin du fichier \textbf{main.o}. Dans notre cas, les fichiers \textbf{c.h} et \textbf{d.h} sont déclarés en \textit{header}, ce qui implique que les fichiers \textbf{c.o} et \textbf{d.o} sont aussi demandés. De plus, les fichiers \textbf{a.h} et \textbf{b.h} sont demandés par \textbf{c.h}, et \textbf{a.h} est demandé par \textbf{d.h}. Pour compiler le fichier \textbf{main.c}, il faut donc avoir accès aux fichiers \textbf{main.o}, \textbf{a.o}, \textbf{b.o}, \textbf{c.o} et \textbf{d.o}.
On ajoute donc les lignes suivantes au fichier \textbf{Makefile} :
\begin{verbatim}
main: main.o a.o b.o c.o d.o
    gcc -o main main.o a.o b.o c.o d.o
\end{verbatim}

Pour le reste, on construit alors un graphe de dépendance (voir figure \ref{fig:dependance_test}), où $\mathbf{x}\rightarrow \mathbf{y}$ signifie que le fichier \textbf{y} a besoin du fichier \textbf{x} pour être compilé.


\begin{figure}[!h]
\centering
% https://tikzcd.yichuanshen.de/#N4Igdg9gJgpgziAXAbVABwnAlgFyxMJZABgBpiBdUkANwEMAbAVxiRAB12cYAPHf4FAB0AYwC+IMaXSZc+QijIBGKrUYs2nbnwHCAFhKkzseAkTIAmVfWatEHLr345gdUYekgMJ+edIBma3U7B21nVyEDSU9vOTNFUgAWINtNRx0XEXdo4ziFElIAVhSNey0nASyooy9ZU3yyADYSkPKM4AAjbJrY+r8Adha0sIEu6pi63xQLUhVqG1LQipdhCA9cvumAobL08Lc1nNqfeOQZ4vng4eXgLMOeydOZwcvU3ZGXLvuJk-zEpJ2S3aAFs6FgwOtjnkiP8rK9Fm1wqDwUJ7qoYFAAObwIigABmACcIMCkGQQDgIEgABw1QnEpBKagU6m0okkxAzcmUxAATlZ9MQ-iZ3L5njp7P+XIZxH57MKwulsqQjQViCUSiViH6qvVmqFUrVZIY4JCcAgxqgR3FSH1zMQVOoHRgYEtiBlYrZ1J1Fk1PO9mvV-o9AqUZLtSh9wfZSlt3Ij1D0MDorrATAYDE18oNMcdztd7vxnrVOv1TpdpLEFDEQA
\begin{tikzcd}
\textbf{d.c} \arrow[rrd]                          &  &                            &  &                            \\
\textbf{d.h} \arrow[rr] \arrow[rrrrd]             &  & \textbf{d.o} \arrow[rrddd] &  &                            \\
\textbf{a.c} \arrow[rrd]                          &  &                            &  & \textbf{main.o} \arrow[dd] \\
\textbf{a.h} \arrow[rr] \arrow[rrdd] \arrow[rruu] &  & \textbf{a.o} \arrow[rrd]   &  &                            \\
\textbf{c.c} \arrow[rrd]                          &  &                            &  & \textbf{main}              \\
\textbf{c.h} \arrow[rr] \arrow[rrrruuu]           &  & \textbf{c.o} \arrow[rru]   &  &                            \\
\textbf{b.c} \arrow[rrd]                          &  &                            &  &                            \\
\textbf{b.h} \arrow[rr]                           &  & \textbf{b.o} \arrow[rruuu] &  &                           
\end{tikzcd}
\caption{Relations de dépendance entre les fichiers du dossier \textbf{test}}
\label{fig:dependance_test}
\end{figure}

Nous pouvons alors en déduire le fichier \textbf{Makefile} (figure \ref{code:makefile_test}).



\begin{figure}[h]
\lstinputlisting[language=c]{ testproj/Makefile }
\caption{Code source de \textbf{Makefile} du projet de test}
\label{code:makefile_test}
\end{figure}


\end{question}


\newpage


\begin{question} %% Question 2

Nous créons un module \textbf{regles}, où l'on définit une structure \texttt{regle}. Une \texttt{regle} contient un pointeur vers une chaîne \texttt{nom}, un pointeur vers un tableau \texttt{prerequis} qui représente l'ensemble des fichiers nécessaires pour compiler [\texttt{nom}]. L'entier \texttt{lenPrerequis} représente la taille de ce tableau. La liste des commandes à effectuer est stockée dans la liste (nous définissons les listes chaînées dans le module \textbf{listeRegles}) des \texttt{commandes}.

Enfin, comme nous avons choisi d'implémenter l'extension qui remplace la date de dernière modification du fichier par un hash, on ajoute les attributs \texttt{lastModified} (temps) et \texttt{hashModified} (booléen), dont le fonctionnement est détaillé en question 10.


Cette structure est munie des fonctions permettant la création de règle (\texttt{createRegle}) et la suppression (\texttt{freeRegle}), en utilisant les fonctions \texttt{malloc} et \texttt{free}. Les autres fonctions permettent la gestion des dernières modifications et des hashs, et sont détaillées en question 10.

\end{question}

\begin{question} %% Question 3

Nous définissons un module \textbf{listeRegles} qui permet de gérer les ensembles de règles. Comme le code nécessite de parcourir l'ensemble , nous décidons d'implémenter ces ensembles à l'aide de listes simplement chaînées.
La structure \texttt{listeRegles} est donc le pointeur \texttt{NULL}, ou un couple (pointeur vers une \texttt{regle}, pointeur vers une \texttt{listeRegles}).

La fonction \texttt{createListeRegle} crée un pointeur vers une liste vide de règles (le pointeur \texttt{NULL}), et \texttt{freeListeRegle} supprime la liste de règles, en itérant la fonction \texttt{freeRegle} sur chaque règle.

La fonction \texttt{addRegle} permet d'ajouter une \texttt{regle} à une \texttt{listeRegles}, via un ajout à la tête.

La fonction \texttt{rechercheRegle} recherche une \texttt{regle} par son nom dans une \texttt{listeRegles}, et la renvoie si elle existe.

La fonction \texttt{iterRegles} permet d'appliquer une fonction à tous les éléments d'une liste de type \texttt{listeRegles}. 

La fonction \texttt{revListRegle} renvoie la liste renversée de celle donnée en argument.

Enfin, les fonctions \texttt{getLatestModify} et \texttt{childModified} ainsi que quelques détails seront traités en question 10.



\end{question}

\begin{question} %% Question 4

Le fichier \textbf{Makefile} du projet est le suivant :

\begin{figure}[!h]
\lstinputlisting{Makefile}
\caption{Code source de \textbf{Makefile} de \texttt{myMake}}
\end{figure}
\end{question}

\begin{question} %% Question 5

Dans le fichier \textbf{main.c}, nous définissons une fonction \texttt{makefile2list}, qui prend en argument un pointeur vers un fichier \textbf{makefile} valide, et renvoie un pointeur vers une liste de règles représentant les règles définies dans le fichier.

Son fonctionnement est décrit dans l'algorithme \ref{alg:cap}, page \pageref{alg:cap}.

\begin{algorithm}[!h]
\caption{ $\quad$ \texttt{makefile2list}}
\label{alg:cap}
\begin{algorithmic}
\State \texttt{L} = \texttt{createListeRegles}
\State \texttt{R} = \texttt{createRegle("", [], 0, [])}
\State \texttt{enregistré} = \texttt{VRAI}
\State \texttt{ligne} = première ligne de \emph{makefile}
\While{il reste des lignes à lire dans \emph{makefile}}
\If{\texttt{ligne = []} \textbf{et} enregistré = \texttt{FAUX}}
	\State \texttt{enregistré} $\gets$ \texttt{VRAI}
	\State \texttt{R.commandes} $\gets$ \texttt{revListCommande(R.commandes)}
	\State \texttt{L} $\gets$ \texttt{addRegle(L,R)}
	\State \texttt{R} $\gets$ \texttt{createRegle("", [], 0, [])}
\ElsIf{\texttt{ligne} est de type \texttt{nom : <listePrerequis>}}
	\State \texttt{enregistré} = \texttt{FAUX}
	\State \texttt{R.nom} $\gets$ \texttt{nom}
	\State \texttt{R.prerequis} $\gets$ \texttt{<listePrerequis>}
	\State \texttt{R.lenPrerequis} $\gets$ |\texttt{<listePrerequis>}|
\ElsIf{\texttt{ligne} de type \texttt{<commande>}}
	\State \texttt{R.commandes} $\gets$ \texttt{ligne::R.commandes}
\Else \Comment{Saut de ligne supplémentaire}
    \State ()
\EndIf
\EndWhile
\If{\texttt{enregistré} = \texttt{FAUX}}
	\State \texttt{R.commandes} $\gets$ \texttt{revListCommande(R.commandes)}
	\State \texttt{L} $\gets$ \texttt{addRegle(L,R)}
\EndIf

\Return L
\end{algorithmic}
\end{algorithm}




La lecture de chaque ligne s'effectue grâce à la fonction \texttt{getline} \cite{getline}. Ainsi, la commande \begin{center}
\texttt{tailleLigne = getline(\&ligne\_buffer, \&tailleLigne\_buffer, makefile)}
\end{center}
copie la ligne courante du fichier \texttt{makefile} dans la chaîne \texttt{ligne\_buffer}, stocke sa longueur dans \texttt{taille\_ligne} (en comptant le retour à la ligne). La variable \texttt{tailleLigne\_buffer} n'est pas utilisée ici.


On détecte si une ligne est de type \texttt{nom : <listePrerequis>}, \texttt{<commandes>} ou \texttt{<$\backslash\!\!$ n>} en mesurant sa longueur. Si la longueur est 1, alors la ligne est un saut de ligne. Sinon, les lignes de commandes commencent par une tabulation.

On scinde la chaîne de caractères de type \texttt{nom : <listePrerequis>} avec la fonction \texttt{strtok} \cite{strtok}. Par exemple, la commande
\begin{center}
\texttt{token = strtok(copyLigne, ":");}
\end{center}
stocke dans la chaîne \texttt{token} les premiers caractères de \texttt{copyLigne} jusqu'à rencontrer un ``:'' (ce caractère ne sera pas inclus dans \texttt{token}). Pour obtenir le bloc suivant, il suffit de faire
\begin{center}
\texttt{token = strtok(NULL, " ");}
\end{center}
pour le stocker dans \texttt{token}. La chaîne source (ici \texttt{copyLigne}) est implicite.


\end{question}

\begin{question} %% Question 6

Pour la fonction \texttt{make\_naive}, nous avons adopté une approche récursive.
\begin{itemize}
	\item \underline{Cas de base} : les fichiers sources (\textbf{.c}) ainsi que les headers (\textbf{.h}) sont des fichiers qui ne nécessitent aucune compilation.
	On peut alors directement finir l'exécution ici.
	\item \underline{Cas général} : On utilise la fonction \texttt{createListeRegleFromPre} qui prend en argument la liste de règles qui a été créée lors de la lecture du \textbf{Makefile}, ainsi qu'une règle,
	et renvoie une liste de règles constituée des prérequis de la règle passée en argument.

	Cependant, en lisant uniquement les règles depuis la liste du \textbf{Makefile}, aucune règle concernant les fichiers \textbf{.h} et \textbf{.c} n'est renvoyée, puisque par définition ils n'ont aucune règle associée.

	On introduit alors le concept de \underline{pseudo-règle}. Il s'agit d'une fausse règle représentant un tel fichier. Il n'est alors pas nécessaire d'initialiser certains attributs comme les prérequis ou les commandes puisqu'il n'y en a pas.
	On crée donc de telles règles lors de l'exécution de \texttt{createListRegleFromPre}, et on prévoit de libérer la mémoire utilisée en même temps que celle prise par la liste retournée.

	Une fois que l'on a construit la liste des prérequis, on effectue les appels récursifs à l'aide de la fonction \texttt{iterRegles} qui va appliquer la fonction \texttt{make\_naive} avec comme premier argument la liste générale de règles du \textbf{Makefile} à chaque prérequis.

	Après ces appels, on a donc construit récursivement les prérequis, et l'on peut ensuite construire la règle elle-même.
	
	On libère ensuite la mémoire prise par la liste des prérequis construite ainsi que les pseudo-règles.
\end{itemize}
\end{question}

\begin{question} %% Question 7
La fonction \texttt{make} est presque identique à la fonction \texttt{make\_naive}. On y ajoute une condition après les appels récursifs pour décider si l'on doit recompiler certains fichiers ou pas.

On récupère la date de modification du fichier liée à la règle considérée, ainsi que la dernière date de modification des prérequis avec la fonction \texttt{getLatestModify}.

On utilise l'attribut \texttt{lastModified} de \texttt{regle} pour stocker ces dates. On l'initialise une première fois lors de la construction de la règle. On utilise la librairie \texttt{stat} pour récupérer les dates dans la fonction \texttt{getLastModified}.

Si l'un des prérequis a été modifié après la règle considérée, cela signifie que le fichier lié n'est pas à jour : il nous faut donc le recompiler ; on exécute les commandes idoines. En plus d'exécuter les commandes, on change aussi la date de dernière modification pour permettre de recompiler en cascade lors de la remontée des appels récursifs.

\end{question}

\begin{question} %% Question 8
Dans la fonction \texttt{main}, on vérifie tout d'abord le nombre d'arguments fournis. Si aucun argument est fourni (à savoir \texttt{argc} = 1), la cible sera la première règle de la liste.
Sinon, on recherche la règle associé au nom passé en argument et on applique la fonction de construction à cette règle.

On libère ensuite la mémoire utilisée par nos ressources, en l'occurence les règles et la liste de règle générale.
\end{question}

\begin{question} %% Question 9
	
\end{question}

\begin{question} %% Question 10 = extensions
	Nous avons décidé d'implémenter la méthode utilisant une fonction de hash pour vérifier s'il y a eu un changement depuis la dernière compilation.

	Après une première tentative peu fructueuse à l'aide d'une fonction de hashage plus simple, nous avons utilisé % Biblio
	une implémentation en SHA-256 en C. La librairie est principalement composée de trois fonctions et une structure.
	\begin{itemize}
		\item On initialise tout d'abord une structure qui sera utilisée par les algorithmes pour stocker les informations nécessaires au calcul du hash.
		\item On lit le fichier \textbf{.hash} qui contient les hash calculés lors de la dernière exécution. On le lit par bloc de 64 bytes et l'on utilise la fonction \texttt{sha256\_update} pour hasher au fur et à mesure.
		\item Une fois que l'on a fini la lecture, on utilise \texttt{sha256\_final} pour calculer le hash de retour.
	\end{itemize}

	Les hash calculés par la fonction SHA-256 sont de 32 bytes. Après les avoir calculés, on les stocke dans le fichier en un nombre hexadécimal de 64 chiffres avec avec \texttt{..., sprintf("\%2x", ...)}

\end{question}


% A rajouter à terme :
% Pointeurs de fonctions en C

\begin{thebibliography}{99}
  \bibitem{getline}
  \textit{Parcourir un fichier texte ligne par ligne en C},
  \texttt{code-examples.net/fr/q/8c794b},
  08/10/22.

  \bibitem{strtok}
  \textit{C library function-strtok()},
  \texttt{\\tutorialspoint.com/c\_standard\_library/c\_function\_strtok.htm},
  08/10/22.
  
  \bibitem{hash}
  \textit{Hash Functions},
  \texttt{cse.yorku.ca/$\sim$oz/hash.html},
  08/10/22.


\end{thebibliography}

\end{document}